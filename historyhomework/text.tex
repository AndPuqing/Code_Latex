\documentclass[a4paper]{article}
\usepackage[margin=1in]{geometry} % 设置边距,符合Word设定
\usepackage{ctex}
\usepackage{lipsum}
\title{\heiti\zihao{3} 为什么说中国共产党是中国人民抗日战争的中流砥柱?}
\author{\songti 梁嘉铭}
\date{2020.11.17}
\begin{document}
\maketitle
\section{政治上}
中国共产党积极倡导、促成和维护抗日民族统一战线。
\par
为了更好地抗战,中国共产党提出了建立抗日民族统一战线的主张。
1935年,在中日民族矛盾已逐步上升为主要矛盾的形势下,
中共中央发表了著名的《八一宣言》,呼吁全国各党派、
各军队、各界同胞“停止内战,以便集中一切国力(人力、物力、财力、武力等)
去为抗日救国的神圣事业而奋斗”。
这标志着中国共产党抗日民族统一战线策略思想的初步形成。
1935年12月,中共中央在瓦窑堡会议上又正式制定了建立抗日民族统一战线的策略方针。
中国共产党不仅是抗日民族统一战线的首倡者,而且为促成它付出了巨大努力。
1936年12月,中共中央积极支持和平解决西安事变,
使这一事件成为时局转换的枢纽,为实现第二次国共合作奠定了基础。
\maketitle
\section{军事上}
中国共产党领导人毛泽东集中全党智慧写成《论持久战》,揭示了抗日战争的发展规律和坚持抗战的战略方针,
对全国抗战的战略指导产生了积极的影响。
\par
为了贯彻执行全面抗战路线,中国共产党实行了开辟敌后战场的战略决定。
其与国民党军队在正面战场的作战形成了有利配合。到1944年春季,
敌后战场人民军队已经抗击着全部日军的64\%。
\par
中国共产党人民军队提出的游击战被提到了战略地位,其牵制住了日军的进攻,
减轻正面战场压力。在实行战略反攻准备时,就是在人民军队抗日根据地展开的。
\maketitle
\section{文化以及建设上}
通过根据地军队战胜困难,农业生产和工商业都得到了恢复和发展,
为坚持抗战,争取胜利奠定了物质基础。
\par
中国共产党在国民党统治区进行抗日民主运动和进步文化工作,
是全民族抗日战争中的一条重要的战线,对于激发大后方的爱国民主意识,
坚持国共合作团结抗战,支援抗战前线,积蓄革命力量等发挥了重要作用。
\maketitle
\section{总结}
中国共产党在抗日战争中起到了中流砥柱作用。
具体表现在政治,军事,文化上,中国共产党以有限的资源做出了巨大的贡献,
反观国民党浪费严重,贡献与资源不符。中国共产党在抗日战争中,用不足国民党四十分之一的资源在环境更险恶的
敌后牵制了一半以上的在华日军以及几乎全部伪军,付出不足全国六分之一的伤
亡,造成了在华日军三分之一强的战损,并消灭了数量更多的伪军。在敌后作战
十几万次,利用从日军手中抢回的资源发展成为超过一百三十万的大军,解放了
一百多万平方公里的土地和一亿多民众,越战越强。中国共产党为抗日战争所付出的代价是人民所看到的,
为坚持抗战,夺取胜利作出了永远光辉史册的贡献。
\end{document}