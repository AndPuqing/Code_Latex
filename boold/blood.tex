\documentclass[a4paper]{article}
\usepackage[margin=1in]{geometry} % 设置边距,符合Word设定
\usepackage{ctex}
\usepackage{lipsum}
\title{\heiti\zihao{3} 血战湘江观后感}
\author{\songti 2000510129梁嘉铭}
\date{2020.11.18}
\begin{document}
\maketitle
\section{内容概述}
该片以长征史上最为惨烈的湘江之战为背景,
讲述了82年前中国工农红军付出巨大牺牲血战湘江、
不畏艰险以必胜信念保卫党中央的悲壮故事。
影片塑造了以
\textbf{毛泽东}为代表的中国工农红军领导人
在长征中不屈不挠的光辉形象,
深刻的揭示了左倾错误路线给红军带来的惨痛损失,
生动的再现了红军34师官兵们前赴后继,
为苏维埃流尽最后一滴血的大无畏革命精神。


电影真实再现了湘江战役炮火连天、
血流成河的壮烈场面。令人震撼的战争场面,
也展现了红军将士满怀对中国革命的无限忠诚,
满怀对中国共产党的坚定信念;
展现了红军将士在巨大的困难和强大的敌人面前、
不怕任何艰难险阻,不惜付出一切牺牲,
坚决执行命令的革命英勇精神。
\maketitle
\section{感人故事,令人动容}
老兵林裁缝家里的四个儿子都参加了红军,
他每天的工作就是为红军战士一针一线的缝军装,
为红军缝了一辈子的军装。
毛主席和林裁缝的对话,
不仅把领导人对战士的关心表现得淋漓尽致,
还展现出了红军战士顾全大局、勇于牺牲的精神。
林裁缝的儿子林有国血战中歼敌无数,
中弹后与给他当枪架的父亲林裁缝一起倒地,
在中弹后还不忘给父亲整好军帽,
两人为是红军的一员而感到骄傲。


红军师长陈树湘,带领全师官兵奋战在掩护红军撤退的第一线,
英勇就义、誓死不当俘虏。
当他身受重伤、肠子流出时,竟亲手扯断肠子,
誓死不当俘虏,捍卫红军尊严……


当沿途老百姓给红军战士送粮时,红军战士开始一概拒绝,
毛泽东不忍心辜负老乡心意,同意战士们象征性地接受粮食,
但必须给老乡们留下银元。
这就是老百姓的战士,这就是与人民血肉相连的红军,
中央红军当年正是依靠“不拿群众一针一线”等三大纪律、八项注意,
赢得了老百姓的信任,赢得了民心,
最终赢得了抗日战争、解放战争的胜利。


这一幕幕故事,无不让人动容。
感动于“林裁缝”、“陈师长”这些普通红军战士中不平凡的举动,
感动于红军战士中家国大义面前的政治信仰,
感动于毛泽东同志呼吁红军战士“不拿群众一针一线、
留下银元换取老百姓食物和礼物”的为民情怀。
\maketitle
\section{剧情丰富,引人深思}
影片中,大战在即,彭德怀给指战员作战前动员的场景撼人心魄。
“现在是危急时刻,没有退路,我们必须英勇献身!”
上万将士齐呼:“誓死保卫党中央过江。”


陈树湘要求大家:“只剩下一个人,也要与阵地共存亡。”
大家高喊:“为苏维埃新中国流尽最后一滴血!”


这些话语给我极大的震撼与感动,更有启迪,
这就是伟大的长征精神和忠贞信仰,
在任何时代都能焕发出强大的生命力。


博古、李德只知纸上谈兵,使部队陷入蒋介石的圈套,
辎重大搬家的决策使红军行动缓慢,
选择走大路致使部队在敌机轰炸中伤亡惨重。
毛泽东说:“战士抬辎重走,走得慢还累吐血。”
博古说:“那就加快步伐。”朱德说:“走大路,遇轰炸,伤亡严重。”
李德则说:“红军不是胆小鬼。”
毛泽东质疑李德、博古的决策,
博古竟然说:“李德是共产国际派来的,他能错吗?”
这些对话,
揭示了比一个个教条主义行动更可怕的是其背后的教条主义思维方式,
也使对教条主义的批判变得锋利而具有现实的启示意义。


这些对话让我不由得深思:
到底是什么促使这些革命先烈前仆后继、流血牺牲、奋战在革命前线?
到底是什么促使历时两年多、两万五千里的长征完满完成?
也许,这便是信仰的力量。
正是心中有了对政治立场的坚定之情、对心中信仰的虔诚之心,
他们才可以一步步的走下去。
今天所有的一切,都是我们的老前辈、老先烈,他们在那个苦难的年代,
经过了多少的战斗,献出了多少年轻的生命。
在现在没有战争的时代,我们更应该珍惜现在美好的生活,
从自己做起,做好自己的工作,
服务人民,服务社会。
我们更要团结在习总书记为首的党中央周围,
坚定信念,为我们的国家今后更加昌盛做点事。
\end{document}